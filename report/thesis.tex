\documentclass[a4paper, 12pt]{report}
\usepackage[english]{babel}
\usepackage{blindtext}
\usepackage[a4paper, inner=1.5cm, outer=3cm, top=3cm, bottom=3cm, bindingoffset=1cm]{geometry}
\usepackage{fancyhdr}
\fancyhf{}
\fancyhead[LE]{\leftmark} 
\fancyhead[RO]{\nouppercase{\rightmark}} 
\fancyfoot[C]{\thepage}
\pagestyle{fancy}
\usepackage{graphicx}
\graphicspath{ {report/images/} }

\begin{document}

\begin{titlepage}
    \begin{center}
        \Huge
        \textbf{Using Linear Programming to Solve Vehicle Routing Problem}

        \vspace{1.5cm}

        \includegraphics[width=0.4\textwidth]{ucl-logo.jpg}
        
        \vspace{1.5cm}
        
        \Large   
        Muhammad Rafdi\\

        \vspace{0.8cm}

        Department of Computer Science\\
        University College London\\
        London\\
        29 April 2016

        \vspace{0.8cm}
        Supervised by:\\
        Dr Daniel Hulme       
        
        \vfill
                
        
        \normalsize   
        This report is submitted as part requirement for the BSc Degree in Computer Science at UCL. It is substantially the result of my own work except where explicitly indicated in the text. The report may be freely copied and distributed provided the source is explicitly acknowledged.

        
    \end{center}
\end{titlepage}

\newgeometry{inner=1.5cm, outer=3cm, top=2cm, bottom=3cm, bindingoffset=1cm}

\begin{abstract}
\centering
   This project examine the state of the art of linear programming and how to formulate combinatorial optimisation problems in terms of linear programs. We examine a specific problem : The vehicle routing problem and solve a few instances of it based on a given data set using different solvers. The results are then studied thoroughly.
\end{abstract}

\renewcommand{\abstractname}{Acknowledgements}
\begin{abstract}
\centering
 Untuk Mama dan Papa
\end{abstract}

\tableofcontents

% chapter 1
\chapter{Introduction}
Insert intro here talk about a bit about linear programming (in simple terms), perhaps its history, how important it is to how it can be used to solve combinatorial optimisation problems (especially in the field of operations research) such as the travelling salesman problem and of course the vehicle routing problem.

Insert intro here talk about a bit about linear programming (in simple terms), perhaps its history, how important it is to how it can be used to solve combinatorial optimisation problems (especially in the field of operations research) such as the travelling salesman problem and of course the vehicle routing problem.

\section{Project Motivation}
Operations research is exciting new field that is created from Linear programming. Advancement in computer science allows computationally advance method to perform or research to solve combinatorial problems using fast computers that allows problems that are too tedious to compute by hand. It is very exciting interdisciplinary field that combines mathematics and computer science.  Would like to experience what OR analyst solve on day to day basis etc. Would like to apply mathematical models that have been developed for years to solve real world scenarios. Also maybe teach how people to make mathematical model?

\section{Aims and Scope of Project}
1) Thorough understanding linear programming as a method to solve combinatorial optimisation problems
2) Model a vehicle routing problem based on real world scenario and dataset
3) Obtain the optimized results (the minimum distance and the paths).
4) Compare and contrast different tools used to get the results

Will not attempt to create new algorithms or implement of linear programming solver!

\section{Methodology}
Understand the given problem and model it in terms of linear program. Model the linear program using the given tools and solve it. Verify the model is privides the correct solution by using a test data set in which the correct solution is known in advance. Compare and contrast the results obtained from different tools.

Understand the given problem and model it in terms of linear program. Model the linear program using the given tools and solve it. Verify the model is privides the correct solution by using a test data set in which the correct solution is known in advance. Compare and contrast the results obtained from different tools.

% chapter 2
\chapter{Vehicle Routing Problem}
In this section I will talk about vehicle routing problem
\section{Definition}
First introduce the classic travelling salesperson problem, describe the entities (edges, nodes cost, subtour elimination, objective function etc) that are present within the problem. Then add more constraints to formulate the VRP problem. Also mention the equations.

\section{Specification}
In this section, describe the problem of the given dataset. Given 227 nodes including depot, we want to formulate the capacitated routing problem with each vehicle visiting at most K number of nodes. Capacity don’t really matter, assume that demands are all the same. Make all demands for all cities the same and capacity to the number of cities the node can visit. Also introduce time window variable. Perhaps can formulate time window like  capacity constraint, or perhaps model time window in terms of the times in which the vehicle must arrive.

\section{Variants}

\section{Computational Complexity}

just a simple section. This would be your report. Learn latex properly okay. Loreme ipsum dolor amet. Hahahah good one. CS is fun!
%this is a comment you won't see it in your pdf!

New paragraph! This is exciting. Learn latex properly okay. Loreme ipsum dolor amet. Hahahah good one. CS is fun! \LaTeX\ and \TeX\ is fun!

% chapter 3
\chapter{Theory}
just a simple section. This would be your report. Learn latex properly okay. Loreme ipsum dolor amet. Hahahah good one. CS is fun!

New paragraph! This is exciting. Learn latex properly okay. Loreme ipsum dolor amet. Hahahah good one. CS is fun! \LaTeX\ and \TeX\ is fun!
\section{Linear Programming}

\section{Integer Programming}

\section{Heuristics}

\section{Constraints Programming}


% chapter 4
\chapter{Implementation}

\section{Tools}

\section{Data sets}

\section{Optimisation Model}
\subsection{Sets, Parameters and Variables}
\subsection{Capacitated Vehicle Routing Problem Model}
\subsection{Capacitated Vehicle Routing Problem with Time W Model}

% chapter 5
\chapter{Results}


% chapter 6
\chapter{Conclusion}

\end{document}

