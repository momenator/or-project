% chapter 4
\chapter{Implementation}

This chapter contains the bulk of the analysis of the VRP problem. In this chapter We will discus the implementation and testing of the model
using the chosen tools.

\section{Tools}
In this project, we will be using LP tools: Gurobi, Google Optimization Tools, Optaplanner to solve the problem
defined in the previus chapter. These tools have their own unique APIs and input format. Each of them
have different implementation of the LP solver that causes them to produce different results given the same input,
as we shall see in the next chapter.

Gurobi is a optimisation tool built for solving LP based problems. Its LP solver is written in C and it comes with
APIs to port many different programming languages including Java, C++, Python and a few others. Gurobi allows you
to build any models for any LP problem, giving users full control to implement any algorithms or heuristics that they
prefer. It claims to be the fastest solver amongst 3 other open source solvers\footnote{refer to
this link}, none of which are used in this project. Gurobi is one of the most expensive commercial LP solver in the industry
and its used by many corporations such as FedEx, Netflix and Google. Academic licenses is also available for universities
and its affiliated individuals for free.

or-tools by Google is an optimisation suite for solving various optimisation problems, including VRP. It contains a constraint programming
solver, unified interface for other solvers (e.g Gurobi, GLPK, CPLEX, etc), implemented mostly in C++. Like Gurobi, it also comes with APIs
to support other major programming languages, albeit less variety. This tool allows to focus on modelling the problem at hand, without worrying
too much about the algorithms and the heuristics, as they have been implemented and packaged with the solver. It is an open souce
software that is used in internally at Google that offers various advantages such as: high quality, portability, has active user community.

Optaplanner is an constraint programming engine built in Java for solving optimisation problems. Unlike the other two solvers, it does not
have API to support other languages. In addition, it takes the input in the form of XML file, which are then processed by the engine. It has
predefined XML tags used to model various problems and built-in implementation of algorithms and heuristics for solving them. What Optaplanner
lacks in portability, it makes it up in usability. The XML input format allows users to define the problem rather than implementing the procedures
to solve the problem. In addition to usability, it comes with a nice GUI that visualise common optimisation, including the VRP.

These tools are run on the same hardware. The processor of this hardware is 2.8GHz dual-core Intel i5. It has
8GB 1600 MHz DDR3 memory and is running OSX version 10.9.5.

\section{Datasets}
We have prepared 3 datasets for this project. The first one is a benchmark dataset with known solution to test the tools
if they can find the optimal solution in a small instance. The next dataset is used to compare the compare the performance
of the tools under a time limit. The last datasets are the dataset of the actual problem given by the company. All datasets
share the same schema: Node number, Latitude and Longitude.

The details of the 3 datasets are tabulated in the table below:
\begin{table}[!ht]
    \begin{center}
        \begin{tabular}{ | l | l | p{10.5cm} |}
        \hline
        Name & Size  & Description \\ \hline
        CVRP-9 & 9 nodes  & This is benchmark dataset that is used to build a simple CVRP model to test
        if the LP tools can produce an optimal solution that are close to the known solution under a small load. The vertices in the datasets
        are arranged in such a way that the optimal routes are obvious and can be obtained without calculation.\\ \hline
        CVRP-60 & 60 nodes & This is a dataset that is used to model a CVRP model that is used to compare the performance of tools
        under a time limit of 5 minutes. We take the results generated and compare their optimality.\\ \hline
        CVRP-227 & 227 nodes & This dataset is the actual problem dataset from the company. It is used to model both CVRP and CVRPTW models
        in this project. This dataset can be used to model CVRPTW model even without time windows data because the time window variables are uniform across
        all customers. One of the main objective of this project to collect the optimal distance and its respective routes.\\
        \hline
        \end{tabular}
        \caption{Datasets description}
        \label{table:dataset_description}
    \end{center}
\end{table}

\section{Model Parameters}
The model parameters are variables that is For CVRP-9 and CVRP-60, the parameters are similar
Sets, Parameters and Variables
Capacitated Vehicle Routing Problem Model
Capacitated Vehicle Routing Problem with Time Window Model

\section{Model Implementations}
Talk about the implementation, create model with both python script and the file (Gurobi and OR tools), or just
input file (Optaplanner)
