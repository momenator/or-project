% chapter 1
\chapter{Introduction}
Linear Programming is a mathematical method to find an optimal value of a function in a linear system under a set of
constraints. This method has been around since the 19th century, but it was not fully developed until the World War 2,
where it was used for resources planning. Since its conception, this method has been widely used to solve problems
in Applied Mathematics and Operations Research (OR). Along with the advancement of computing, companies can now increase their
decision making capabilities with efficient technique of linear programming and significant computational power.

Linear programming is used to model to many combinatorial problems, such as vehicle routing.
The vehicle routing problem is a common problem that is commonly observed in logistics.
It is a special instance of the famous traveling salesperson problem in which the tour of
all 'cities' in the given graph into several parts, subject the availability of resources.

In this project, we will conduct an Operations Research study by using linear programming to find the optimal routes of an instance of a vehicle routing
problem. We will use three linear programming tools to arrive at a solution and compare and contrast the results of each one of them.

\section{Project Motivation}
Linear programming has allowed organisations to optimise many aspects its operations. In many occasions, the problem involves
finding the optimal combination of input that yields the greatest output. Amongst the many problems,
vehicle routing is one of them. Many studies have been carried out on this problem due to its role in minimising operational
costs of companies, especially in their logistics department. Logistic plays a significant role in many businesses and it is one of the major
driving force of the economy. In the UK alone, Logistics and Posts Sector is worth approximately
\pounds55bn to the economy and comprises 5 percent of the GDP\footnote{According to a logistics report by PWC }.
In the US, the cost attributed to logistics rose up to US\$1.45 trillion in 2014, an increase of 3.1 percent from the previous year.
These figures are likely to increase given the increasingly globalised economy. Thus, minimising operational costs will not only make companies
more competitive, but it will also increase their profits.

In this project, we want to simulate a scenario where we can optimise
a problem to minimise cost in a fictional delivery company DFFS. We will demonstrate how optimisation can help them solve a vehicle
routing they face.

Aside from financial benefits, the vehicle routing problem is an interesting problem to the mathematically and computationally apt individuals.
There are wide variety of algorithms that can be used to approximate the optimal solution.

A study on the comparison of solvers has been done before in this paper. However, it does not include other open souce LP tools
such as or-tools and Optaplanner, both of which contains their own unique engine to solve LP problems. It will be useful to see the comparison
of different tools as it will help OR analysts to decide which tool is the best for them given their current needs. We have chosen the 3 tools mentioned
because they are relatively popular tools used in the industry there that is relatively well documented and
has an easy to use APIs in various programming languages including Python, the language that we use in this project. Most importantly, they have not
been compared alongside one another.

\section{Goals and Scope of Project}
We have identified a few goals to benchmark this project and a scope to limit the discussions that may have been related to this project. The goals are as
follows:
\begin{enumerate}
\item To thoroughly understand linear programming method to solve optimisation problems.
\item To build a mathematical model of the vehicle routing problem based on the given dataset.
\item Obtain the optimised results (the minimum distance and the paths) using the chosen tools.
\item Compare and contrast the results and the usage of those tools.
\end{enumerate}

Creating novel algorithms for optimisation problems is hard and requires years of experience in Algorithms.
Implementing linear programming solver is also difficult and it requires very high level of software engineering expertise, in addition to
the algorithmic knowledge mentioned. For these reasons, we will not be attempting to create new algorithms or implement linear programming solvers to solve
the vehicle routing problem. Instead we will model the VRP instance using softwares and algorithms that are already available.
In addition, other methods that can be used to solve this problem such as \textit{genetic programming} and \textit{dynamic programming} will
not be covered in this project.

\section{Methodology}
Peforming the calculations for linear programming is not as hard as it used to be given the availability of computing power that we have today.
In this project we will be using three applications: Gurobi, OR Tools by Google and Optaplanner to retrieve the optimal solution. Each of them
contains their own implementation of \textbf{linear programming solver}, a program that can solve linear programming problems. We model them
in a form that can be solved by these solvers and used the solvers to obtain the solution.

This project will be carried out in seven steps, as per the steps taken in a typical OR study:

\begin{enumerate}
\item Define the problem in a given system and formulate it into a linear program. A linear program is a problem that can be solved
using linear programming. In this step, the objective and the constraints of the system is studied thoroughly to ensure
that the program models the given problem correctly.
\item Collect the data that represents the problem at hand and estimated the parameters that are required to produce
the optimal result.
\item Create the mathematical models of the given problem. We will use standard linear programming notation to build these models. This is
also the step in which we implement the linear program using the chosen softwares.
\item Verify that the model is correct. We achieve this by running the implemented linear program through a benchmark dataset with known solution.
\item Select suitable alternative. We may not always get the desired answer, due various reasons such as limited timescale and computing resources.
In this case, we want to select an alternative objectives that will be useful to the parties involved.
\item Run the given problem in the models that we have build and record their result and performance. We may need to go back to step 1 if the results obtained
are not satisfactory.
\item Provide recommendations to the parties involved based on the results obtained.
\end{enumerate}

For comparing the performance of different linear programming tools, we use a 60 node CVRP model. Time limit of 5 minutes is imposed
when running the model on all tools. The optimal distances obtained using each tools will be recorded and compared. In addition, the usage
of the tools will also be discussed.

We use a lot of terminologies interchangeably here such as blah blah blah. This serves as a heads up to the reader... more explanation needed here..

\section{Outline}
In chapter 2, we will elaborate the underlying theory behind linear programming and its state of the art.
The vehicle routing problem will be discussed in detail in chapter 3. The implentation of linear programming to solve the vehicle routing problem will be
discussed in chapter 4, followed by the results in chapter 5. Finally, I will conclude the results and identify potential future works
of this project in chapter 6.