% chapter 1
\chapter{Introduction}
Linear Programming (LP) is a method to find the optimal allocation of resources among competing activities
in a system of linear equations \cite{APMBradley}. There had been many methods of solving linear equalities
problems since the 19th century, none of which made any significant impact. The turning point came in 1947, where George Dantzig
developed the Simplex method \cite{wiki:SA}, an algorithm to solve LP based problems that was highly influential in the development and practice of science and engineering such that it was
listed as one of the top 10 algorithms of the 20th century. In addition, it was also highly influential in the creation
 of Operations Research (OR), a field of study that concerns with making better decisions through mathematical analysis \cite{wiki:OR}.
 With increasingly higher computational power and
the efficiency of the simplex method, LP has become the most powerful optimisation method that enchance decision making
process.

LP can be used to model to many combinatorial problems, such as vehicle routing.
The vehicle routing problem (VRP) is a special instance of the traveling salesman problem(TSP) \cite{Dantzig1959} whereby the objective
is to find the routes for a fleet of vehicles to satisfy the demands of customers that yields the minimum distance.

In this project, we shall act as an OR consultant for a Furnish Ltd\footnote{A fictional company}, a leading UK furniture store to solve their vehicle
routing issues and provide an analysis on popular LP optimisation tools.

\section{Project Motivation}
Logistics plays a significant role in many businesses and it is one of the major
driving force of economies around the globe. In the UK alone, Logistics and Posts Sector is worth approximately
\pounds55bn to the economy and comprises 5 percent of the GDP\footnote{According to recent report by PWC. Link: \url{http://goo.gl/iDkTe3}}.
In the US, the cost attributed to logistics rose up to US\$1.45 trillion in 2014, an increase of 3.1 percent from the previous year.\footnote{\url{http://goo.gl/giDPjE}}
These figures are likely to increase given the increasingly globalised economy. Thus, minimising operational costs will not only make companies
more competitive, but it will also maximise their revenue.

Furnish delivers furnitures to various locations around UK. It currently does not have vehicle routing solutions for their deliveries, which
have resulted in higher operational costs. They are looking to cut cost by trying to implement a computational solution
to solve their routing problem. In addition, they are looking to open a new department dedicated
to solving their routing problems, and would like some advice on getting started. By solving the problems
given by Furnish, it will bring insights to the reader as to what OR analysts do in their job. It will also show
the reader the process of implementing the LP formulations based on a given problem, which has not been widely documented.

This project also contains an analysis on three LP tools: Gurobi, or-tools and Optaplanner. There have been many studies
that analyses the performance of LP solvers \cite{Meindl2012,gurobi:solvers,Hakan2012}, all of which did not include or-tools
and optaplanner. Analysing the performance of or-tools and Optaplanner will bring new perspectives on the tools that are best
for solving a certain instance of a vehicle routing problem.

Why use linear programming for VRP: Because it requires minimal mathematical background to unde etc etc blah blah
and more explanation heree...

The vehicle routing problem has been studied for decades due to its
relevance in applied mathamatics and wide range of applications, especially in logistics. It is also part of a group of problems
that are studied to solve the P vs NP problem\footnote{See \url{http://www.claymath.org/millennium-problems/p-vs-np-problem}}, a major unsolved problem in computer science today.
It has attracted mathematician due to its simplicity of the problem and yet difficulty in finding an optimal algorithm etc etc.. TSP book page 17
TSP book page 37...

\section{Related Works}
A few studies on the comparison of constraint programming solvers have been done before \cite{Meindl2012,Hakan2012}. However, they did not include some open souce LP tools
such as or-tools and Optaplanner. Gurobi Inc\footnote{See \url{http://gurobi.com}} also have carried out comparative study
on its tools against other solvers \cite{gurobi:solvers}. The results of these studies have shown that Gurobi outperforms the other solvers mentioned.


\section{Goals and Scope of Project}
We have identified a few goals to assess the quality of this project and a scope to limit the discussions that may deviate the goals.

The goals of this project are as follows:
\begin{enumerate}
\item To thoroughly understand linear programming and other related concepts to solve problems.
\item To build a mathematical model of the vehicle routing problem based on the given dataset.
\item Analyse the performance and the usage of the chosen LP tools.
\item Obtain the optimal solution (the minimum distance and the routes) using the chosen tool.
\end{enumerate}

Creating novel algorithms for optimisation problems is hard and requires years of experience in algorithms. Furthermore,
implementing linear programming solver is also difficult and it requires very high level of software engineering expertise, in addition to
the algorithmic knowledge mentioned. For these reasons, we will not be attempting to create new algorithms or implementing LP solvers to solve
the vehicle routing problem. Instead, we will model the VRP instance using softwares and algorithms that are already available.
Other methods that are not related to LP, such as genetic programming and dynamic programming will not be discussed in this project.

\section{Methodology}
Peforming the calculations for linear programming is not as hard as it used to be given the availability of computing power that we have today.
In this project we will be using three applications: Gurobi, OR Tools by Google and Optaplanner. Each of them
contains their own implementation of a solver, a program that can solve linear programming problems. We model them
in a form that can be solved by these solvers and used the solvers to obtain the solution. More explanation on this is found on chapter 4.

This project will be carried out in seven steps, as per the steps taken in a typical OR study \cite{Sottinen2009}:

\begin{enumerate}
\item Define the problem in a given system and formulate it into a linear program. In this step, the objective and the constraints of the system is studied thoroughly to ensure
that the program models the given problem correctly.
\item Collect the data that represents the problem at hand and estimated the parameters that are required to produce
the optimal result.
\item Create the mathematical models of the given problem. We will use standard LP notation to build these models. This is
also the step in which we implement the linear program using the chosen softwares.
\item Verify that the model is correct. We achieve this by running the implemented linear program through a benchmark dataset with known solution.
\item Select suitable alternative. We may not always get the desired answer, due various reasons such as limited timescale and computing resources.
In this case, we want to select an alternative objectives that will be useful to the parties involved. By this point, we should have identified all of the client's requirements.
\item We run the given problem in the models that we have build and record their result and performance. We may need to go back to step 1 if the results obtained
are not satisfactory.
\item Provide recommendations based on our analysis to the parties involved.
\end{enumerate}

There are two parts to this project: performance analysis on LP tools and route optimisation. We analyse the performance of the tools on
a benchmark datasets from Augerat et al \cite{Augerat1998}. We also analyse two of our own datasets to determine which tool more suitable to
solve the VRP instance based on the problem given. Based on the outcome of this analysis, we then choose a suitable tool
to use to implement the LP formulations of the given problem.

\section{Terminologies}
There are a few terminologies that are mentioned without explicitly stating what it means and used interchangeably.
For clarity, the following list contains most of the terminologies used in this project along with their respective meaning:
\begin{itemize}
\item Linear/integer Program - An instance of a problem that can be solved using a linear/integer programming method.
\item Formulation - An instance of LP that describes a specific problem.
\item Model (Noun) - An implementation of an LP formulation in LP tools. It
\item Solver - A program that is used to solve LP problems that is based on constraint programming\footnote{refer to chapter 2.3}.
\item Method - Often used interchangeably with algorithm.
\end{itemize}

\section{Outline}
We have introduced the reader to this project, describe its goals and how to steps taken to attain them. In chapter 2,
we will provide some contexts for the reader on the theories that we apply in the project. The following will focus on
analysing the given problem, including forming the correct LP formulation and identifying the client's requirements.
The implentation of LP formulations using the chosen tools will be discussed in chapter 4, followed by the results and
some their comments in chapter 5. Finally, we will conclude the results and identify potential future works
of this project in chapter 6.